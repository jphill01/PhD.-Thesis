\linespread{1.0}

\chapter{Conclusion and Future Directions}
%--------------------------------------------------------- Sec
\section{Thesis Summary} \label{sec:intro1}

The overarching goal of the present thesis was to develop a statistical method to assess intraspecific haplotype sampling completeness using haplotype accumulation curves. The result was a nonparametric stochastic local search optimization algorithm, called {\tt HACSim} (\textbf{H}aplotype \textbf{A}ccumulation \textbf{C}urve \textbf{Sim}ulator), to estimate likely required specimen sample sizes for characterization of species genetic diversity. The method specifically employs Monte Carlo sampling to iteratively extrapolate species' haplotype accumulation curves to detect where curves begin to saturate toward an asymptote. {\tt HACSim} works under the assumption that species are both panmictic (\textit{i.e.} randomly-mating) and are drawn from large ideal populations where the effects of genetic drift are negligible (and thus, population structure does not play a major role). Such a method is useful since it can be used as a null model and stopping criterion for specimen sampling in conjunction with factors such as species rarity and research project budget.

Within the preceeding four chapters, the issue of specimen sample size determination for COI DNA barcoding was investigated in detail. COI is very well represented in existing genomic sequence databases, particularly the Barcode of Life Data Systems (BOLD) and GenBank. The present work clearly points to the increased need to incorporate larger sample sizes into current taxon collection efforts. Crude estimates of adequate sample sizes put forth by various researchers typically range from 5-10 specimens per species. However, it is usually the case that only a single or a few specimens can reasonably be sampled due to species rarity and research budget. In fact, there is no ``one size fits all" when it comes to taxon sampling because different taxa exhibit distinct evolutionary and life histories. Further, different genes used in assessing sampling completeness at the species level have varying rates of molecular evolution. Simulation studies have demonstrated that much larger sample sizes (on the order of hundreds to thousands) are likely needed to reliably ascertain intraspecific genetic variation. Although it would be ideal to sample individuals of a species across their entire geographical or ecological range, as well as across a variety of molecular genetic loci, such a scheme is hindered greatly by factors such as project costs. Many of the issues are highlighted and investigated in Chapter 2 of this thesis.

A much better approach to addressing the question of appropriate sample sizes for DNA barcoding is to instead sample an \textit{optimal} number of specimens of a given species of interest. Chapter 3 presents {\tt HACSim}, a stochastic optimization algorithm for determining necessary specimen sample sizes based on saturation observed in plotted haplotype \\ accumulation curves. The model makes a number of simplifying assumptions that are likely to be violated in real species groups, such as the assumption of panmixia across species ranges. Further, {\tt HACSim} assumes that genetic variation observed in barcode sequences is biologically real. The presence of various sequence artifacts such as insertion/deletions (indels) and nuclear-mitochondrial inserts (NUMTs)/pseudogenes, as well as \\ amplification/sequencing errors (\textit{e.g.}, incorrect base calls) within user-curated databases likely means that necessary sample size are actually much larger than currently predicted.  A detailed simulation study outlined in Chapter 4 shows (1) that {\tt HACSim} suggests sample sizes higher than those routinely employed in practice within DNA barcoding studies and (2) that {\tt HACSim} consistently captures desired levels of within-species haplotype diversity across all wide-ranging sample sizes for a variety of animal species of direct societal value. Based on simulation study results, {\tt HACSim} is shown to have good statistical properties such as high coverage probabilities. Despite this, the assumption of representative sampling of species haplotype diversity is quite unrealistic. While {\tt HACSim} is currently available as an R package from CRAN, a stand-alone R Shiny web application has also been developed and will be available soon for those users who either lack confidence in programming within R or those users who are uncertain {\tt HACSim} is ideal for their problem of interest. In both cases, the {\tt HACSim} Shiny app can be employed to run all examples included in Phillips \textit{et al.} \cite{phillips2020hacsim} (Chapter 3) and Chapter 4, plus other custom simulations of a user's choosing. The app will be accessible in two ways: (1) through the Shiny server, or (2) through the HACSim R package itself via the {\tt launchApp()} function. Future work should investigate the further relaxing of model assumptions so that simulations are made more biologically realistic, in particular the introduction of population structuring. In addition, functionality should be made to allow end users to simulate their own DNA sequence alignments according to various models of nucleotide substitution.

While the present thesis addresses many issues surrounding sample size estimation for DNA barcoding, much still remains to be accomplished on the statistical front.  To this end, Chapter 5 makes a case for the lack of statistical rigor in DNA barcoding as it pertains to accurate estimation of the DNA barcode gap. Arguments revolve around three main areas: (1) improper allocation of specimen sampling effort required to adequately probe levels of standing intraspecific genetic variation, (2) improperly visualizing within- and among-species genetic distances and (3) the inconsistent, inappropriate use, or absence of statistical inferential procedures in DNA barcoding gap analyses. In addressing each of these omissions, simple statistical solutions are offered that are readily accessible to the non-statistician. {\tt HACSim} can be specifically employed to address the issue of \\ disproportionate sampling effort that is evident within BOLD and other genetic data \\ repositories. Kernel density estimation plots, in addition to quadrant plots can greatly aid visualization of DNA barcode sequence data, potentially pointing out failures of DNA barcoding for certain problematic taxa. Lastly, hypothesis testing and resampling via the nonparametric $m$-out-of-$n$ bootstrap can lend strength to both point and interval estimation of intraspecific and interspecfic distances that make up the barcode gap. A case study on fishes of Pacific Canada is used to demostrate problems with current methods and solutions via improved ones.

All species included in data analysis within the present thesis, particularly those of fishes, insects and arachnids for the performance testing of {\tt HACSim}, were selected for two very important reasons. Firstly, these taxa are well represented in BOLD, often comprising hundreds to thousands of sampled specimens. Secondly and finally, all examined taxa possess cultural, regulatory/forensic, medical and/or socioeconomic significance. \\ However, it is important to also consider estimation of species' genetic diversity and \\ determination of adequate specimen sample sizes for taxa besides animals and for genes aside from COI. Herein, it was explicitly decided to focus on solely on animals and COI because sequence records are highly abundant within BOLD. Nevertheless, once \\ community standards improve and further sampling depth is achieved, both plant and fungal species should also be targeted in relation to {\tt HACSim}.  

  

% more sections

