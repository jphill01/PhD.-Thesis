\linespread{1.0}

\chapter{Introduction}
%--------------------------------------------------------- Sec
\subsection{Thesis Overview} \label{sec:intro1}

This thesis outlines a novel statistical framework for assessment of COI DNA barcode haplotype sampling completeness. 

Chapter 2 consists of a literature review of studies conducted to date pertaining to sample size estimation for DNA barcoding, as well as the need for further research in this area. A case study assessing haplotype diversity in the ray-finned fishes \cite{phillips2015exploration} provides motivation for subsequent work presented and examined in the present thesis. 

Chapter 2 details a novel stochastic optimization algorithm caled {\tt HACSim} (short for \textbf{H}aplotype \textbf{A}ccumulation \textbf{C}urve \textbf{Sim}ulator) that is programmed in the R Statistical \\ Environment. {\tt HACSim} can be employed to calculate likely required specimen sample sizes needed to capture a given fraction of genetic diversity within species based on iteratively extrapoling species' haplotype accumulation curves. {\tt HACSim} is applied to various \\ vertebrate and invertebrate taxa of recent interest within the DNA barcoding literature, primarily fishes and arthropods of conservation, medical, and forensic/regulatory \\ importance.

Chapter 3 extends elements discussed in Chapter 2 through delving further into the useability and applicability of {\tt HACSim} via a detailed statistical simulation study to assess both the validity and overall performance of {\tt HACSim} and its utility for assessing \\ intraspecific sampling completeness within DNA barcoding studies for a variety of species mined from the Barcode of Life Data Systems (BOLD) \cite{ratnasingham2007bold}.

Finally, in Chapter 4, it is argued that DNA barcoding is currently lacking in statistical rigor and that better statistical methods are necessary to more accurately assess standing genetic variation at the species level. Proposed approaches are illustrated through a case study focussing on Pacific fishes of Canada \cite{steinke2009dna}.


\subsection{Thesis Statement}

Through the development of a novel stochastic simulation algorithm for the generation of haplotype accumulation curves, the current research will provide a framework that can be employed to determine plausible specimen sample sizes sufficient to quantify levels of haplotypic sampling completeness within species under both uniform and non-uniform haplotype frequency distributions.

% more sections


  
