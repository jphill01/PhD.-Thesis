\linespread{1.0}

\chapter{Thesis Overview}
%--------------------------------------------------------- Sec
\subsection{Thesis Outline} \label{sec:intro1}

This thesis outlines a novel statistical framework for assessment of COI DNA barcode haplotype sampling completeness. Chapter 2 consists of a literature review of studies conducted to date pertaining to sample size estimation for DNA barcoding, as well as the need for further research in this area. Chapter 2 details a stochastic optimization algorithm caled {\tt HACSim} that is programmed in the R Statistical Environment. {\tt HACSim} can be employed to calculate likely required specimen sample sizes needed to capture a given fraction of genetic diversity within species. Chapter 3 is devoted to a small simulation study assessing both the validity and statistical performance of {\tt HACSim}, as well as its overall utility for assessing sampling completeness within DNA barcoding studies. Finally, in Chapter 4, it is argued that DNA barcoding is currently lacking in \\ statistical rigor and that better statistical methods are necessary to more accurately assess standing genetic variation at the species level.


\subsection{Thesis Objectives}


\subsection{Thesis Statement}

Through the development of a novel stochastic simulation algorithm for the generation of haplotype \\ accumulation curves, the current research will provide a framework that can be employed to determine plausible specimen sample sizes sufficient to quantify levels of haplotypic sampling completeness within species under both uniform and non-uniform haplotype frequency distributions.

% more sections


  
