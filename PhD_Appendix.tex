\linespread{1.0}

%--------------------------------------------------------- Sec

\appendix

\chapter{Additional Information Accompanying Chapter 3}

Supplemental information can be found at https://peerj.com/articles/cs-243/. 

\vspace{5mm}

Internal R code for {\tt HACSim} can be found at https://github.com/jphill01/HACSim.R.


\chapter{Additional Information Accompanying Chapter 4}

Supplemental information can be found at https://github.com/jphill01/\\PhD-Thesis-Appendix.

\chapter{Derivation of Approximate Confidence Interval for Sampling Sufficiency ($\theta$) \\ (Equation (3.3))}

The Delta Method permits the approximation of the distribution of a function of a random variable provided said random variable is asymptotically normal.

\vspace{5mm}

Using the univariate Delta Method, an approximate large-sample (asympototic) \\ confidence interval for specimen sampling sufficiency ($\theta$) based on the Central Limit \\ Theorem (CLT) can be derived.  

\vspace{5mm}

For a single parameter of interest, the Delta Method states that

\begin{equation}
\sqrt{n}(g(X_n) - g(\mu)) \rightarrow_d \mathcal{N}(0, [g'(\mu)]^2\sigma^2)
\end{equation}

\noindent where $n$ is the number of observations and $g(X_n)$ is a function of independent and \\ identically distributed (iid) random variables $X_1,...,X_n$ with finite variance $\sigma^2$ \\ approximating some unknown parameter $\mu$.

\vspace{5mm}

For the problem at hand, both $N_i$ and \textit{H}* are constants and $H_i$ is the random variable of interest. In reality, $N_i$ is likely not constant, but is treated as such according to the constant population size assumption implicit in {\tt HACSim}'s underlying model discussed in detail within Chapter 4. For simulation of real species, $N_0$, the initial guess of likely required specimen sample size used to initialize {\tt HACSim}, is known \textit{a priori} since it is equal to the number of specimens/DNA sequences present in the filtered multiple sequence alignment required as input. In addition, the decision to treat $N_i$ as constant avoids the need to use the multivariate Delta Method, which would otherwise require evaluation of several partial derivatives, thus making the mathematics quite cumbersome and tedious.  

\vspace{5mm}

Writing Equation (3.1) as a function of $H_i$ gives

\begin{equation}
g(H_i) = \frac{N_iH^*}{H_i}.
\end{equation}

\vspace{5mm}

Now, differentiating the above expression with respect to $H_i$ using the Quotient Rule, factoring the denominator, and then simplifying leads to

\begin{equation}
g'(H_i) = \frac{-N_iH^*}{H_i^2} = \frac{-1}{H_i}\frac{N_iH^*}{H_i} = \frac{-N^*_{i+1}}{H_i}.
\end{equation}

\vspace{5mm}

By the Delta Method, the estimated variance of \textit{N}* is given by

\begin{equation}
\widehat{Var}[N^*_{i+1}] = [g'(H_i)]^2\hat{\sigma}^2_{H_i} = \left(\frac{-N^*_{i+1}}{H_i}\right)^2\hat{\sigma}^2_{H_i} = \left(\frac{N^*_{i+1}}{H_i}\hat{\sigma}_{H_i}\right)^2.
\end{equation}

\noindent and its corresponding standard deviation and standard error are respectively

\begin{equation}
\widehat{SD}[N^*_{i+1}] = \sqrt{\widehat{Var}[N^*_{i+1}]} = \frac{\hat{\sigma}_{H_i}}{H_i}N^*_{i+1}
\end{equation}

\noindent and

\begin{equation}
\widehat{SE}[N^*_{i+1}] = \frac{\widehat{SD}[N^*_{i+1}]}{\sqrt{N^*_{i+1}}} = \frac{\hat{\sigma}_{H_i}}{H_i}\sqrt{N^*_{i+1}}.
\end{equation}

\vspace{5mm}

Finally, a symmetric CI for $\theta$ can be constructed, leading to Equation (3.3)

\begin{equation}
N^*_{i+1} \pm z_{1-\frac{\alpha}{2}}\widehat{SE}[N^*_{i+1}] = N^*_{i+1} \pm z_{1-\frac{\alpha}{2}}\left(\frac{\hat{\sigma}_{H_i}}{H_i}\sqrt{N^*_{i+1}}\right).
\end{equation}

\noindent where $z_{1-\frac{\alpha}{2}}$ is the $(1-\frac{\alpha}{2})100\%$ quantile of the Standard Normal distribution and $\hat{\sigma}_{H}$ is the estimated standard deviation for the number of haplotypes ($H$) found from sampling \textit{N}* specimens.  However, there are several issues with this approach. First, the resulting interval is symmetric and, while tight, is likely to be biased, as it is centered on the estimated required sample size (\textit{N}*). This is not altogether unexpected due to reliance on the Central Limit Theorem in obtaining an approximate confidence interval for $\theta$. In actuality, \textit{N}* would be expected to fall closer to either the lower or upper endpoint of the contructed CI compared to its centre as species' haplotype accumulation curves begin to saturate toward an asymptote (\textit{i.e.}, \textit{H}*). Second, the resulting confidence interval is calculated from only a single random sample (\textit{i.e.}, a single run of {\tt HACSim}); in reality, any interval estimate should be computed from multiple samples/runs whenever feasible. Methods for producing a non-symmetric interval are currently being investigated. Third, and perhaps most importantly, said confidence interval is unlikely to have desired nominal coverage probability of at least ($1 - \alpha$)100\% (95\% say); a non-symmetric interval may be able to compensate for this discrepancy however. Regardless, calculation of interval estimates for $\theta$ is important given the substantial amount of resources (both cost and effort) required to retrieve additional specimens for a given species of interest.



% more sections

